\question \textbf{Title}

For any positive integers $a$ and $n$, if $d = gcd(a, n)$ (the greatest common divisor of $a$ and $n$),
then

$$ \langle a \rangle  = \langle d \rangle  = {0, d, 2d, . . . , n − d}$$

and thus

$$|\langle a \rangle | = n/d$$
$$(\langle a \rangle  := \left\{a \cdot i\,\text{mod}\,n | i \in N\right\})$$.
Hint: Use Bezout’s lemma. It states that if $a$ and $b$ are nonzero integers with greatest common divisor $d$, then there exist integers $x$ and $y$ such that $ax + by = d$


\begin{solution}
$n/d = k \Leftrightarrow n = k\cdot d$ this also means $\langle d \rangle = \left\{0, 1\cdot d, 2\cdot d,\cdots, (k-1)\cdot d\right\}$. With $(k-1)\cdot d=(n/d-1)\cdot d=n-d$ follows $\langle d \rangle = \left\{0, 1\cdot d, 2\cdot d,\cdots, n-d\right\}$. Now it must still be shown that $\langle a \rangle  = \langle d \rangle$.

If we take any element of the form $a \cdot i \pmod{n}$ and divide it by $d$, the remainder will be the same as if we had divided $a$ by $d$. Therefore, the set of remainders obtained by dividing the elements of $\langle a \rangle$ by $d$ is the same as the set of remainders obtained by dividing $a$ by $d$. This means that $\langle a \rangle = \langle d \rangle$, as claimed in the statement.

\end{solution}
