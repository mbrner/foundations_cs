\question \textbf{Title}

\begin{parts}
\part Question % optional task text
a In a more general max-flow problem, there are many sources and sinks, and we wish
to maximize the total flow from all sources to all sinks. Show how to reduce the more
general problem to the original max-flow problem.
b Assume a flow network with edge and additional vertex capacities. Each vertex v has a
limit on the flow that can pass through it. Explain how to transform this flow network
into an equivalent flow network without vertex capacities.

\begin{solution}
a\\

To maximize the flow from all sources to all sinks we can add another source as well as another sink to the graph, which both have the infinite capacity.

b\\

To get rid of the capacities assigned to vertexes we can just split each vertex, to which a capacity is assigned, into one incoming node and one outgoing node. 
We then link both nodes with an edge to which we assign the capacity of the vertext we just split up 


\end{solution}


\part Question% optional task text


\begin{solution}
Here goes the solution
\end{solution}

\end{parts}


% For tasks without simply remove the \begin{parts}...\part...\end{parts} commands