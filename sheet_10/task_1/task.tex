\question \textbf{Title}

\begin{parts}
\part Question % optional task text
\begin{itemize}
\item Is a Network flow directed or undirected?
directed
\item How is flow defined?
Flow: $f : E \rightarrow R$ + satisfying
1. Flow conservation constraints $\sum_{e:target(e)=v} f (e) =  \sum_ {e:source(e)=v}f (e), for all v \in V \ {s, t}$
2. Capacity constraints
$0 <= f (e) <0 cap(e), for all e \in E$
\item What is the ”flow conservation constraint”?
$\sum_{e:target(e)=v} f (e) =  \sum_ {e:source(e)=v}f (e), for all v \in V \ {s, t}$
\item What is the ”capacity constaint”?
$0 <= f (e) <0 cap(e), for all e \in E$
\item In the formula $\sum_{e:target(e)=v}{f(e)} − \sum_{e:source(e)=v}{f(e)}$, what does target(e) and source(e) stand for?

Source(e) is the node from which the flow starts, i.e. only outgoing edges
Target(e) is the node where the flow end, i.e. only incoming edges and no more outgoing 

\item Do you know what the ”maximum flow problem” is?
max{val(f ) | f is a flow in G}
\item Do you understand what a ”cut” is?
Hopefully, A cut is a partition (S, T ) of V , i.e., T = V \ S
\item What is a ”residual network”?

The residual network G f for a flow f in G = (V , E) indicates the capacity unused by
f . It is defined as follows:
-G f has the same node set as G.
- For every edge e = (v , w) in G, there are up to two edges e 0 and e 00 in G f :
1. if f (e) < cap(e), there is an edge e 0 = (v , w) in G f with residual capacity
r (e 0 ) = cap(e) - f (e).
2. if f (e) > 0, there is an edge e 00 = (w, v ) in G f with residual capacity r (e 00 ) =
f (e).
\end{itemize}

\begin{solution}
Here goes the solution
\end{solution}


\part Question% optional task text


\begin{solution}
Here goes the solution
\end{solution}

\end{parts}


% For tasks without simply remove the \begin{parts}...\part...\end{parts} commands