\question \textbf{Title}

\begin{parts}
\part Is a Network flow directed or undirected?

\begin{solution}
directed
\end{solution}

\part How is flow defined?

\begin{solution}
Flow: $f : E \rightarrow R$ + satisfying
1. Flow conservation constraints $\sum_{e:target(e)=v} f (e) =  \sum_ {e:source(e)=v}f (e), for all v \in V \ {s, t}$
2. Capacity constraints
$0 <= f (e) <0 cap(e), for all e \in E$
\end{solution}

\part What is the "flow conservation constraint"?

\begin{solution}
    $\sum_{e:target(e)=v} f (e) =  \sum_ {e:source(e)=v}f (e), for all v \in V \ {s, t}$
\end{solution}

\part What is the "capacity constaint"?

\begin{solution}
    $0 <= f (e) <0 cap(e), for all e \in E$

\end{solution}


\part In the formula $\sum_{e:target(e)=v}{f(e)} − \sum_{e:source(e)=v}{f(e)}$, what does target(e) and source(e) stand for?

\begin{solution}
Source(e) is the node from which the flow starts, i.e. only outgoing edges

Target(e) is the node where the flow end, i.e. only incoming edges and no more outgoing 
\end{solution}

\part Do you know what the "maximum flow problem" is?

\begin{solution}
max{val(f ) | f is a flow in G}
\end{solution}

\part Do you understand what a "cut" is?

\begin{solution}
    Hopefully, A cut is a partition (S, T ) of V , i.e., T = V \ S
\end{solution}

\part What is a "residual network"?

\begin{solution}

    The residual network G f for a flow f in G = (V , E) indicates the capacity unused by
    f . It is defined as follows:
    -G f has the same node set as G.
    - For every edge e = (v , w) in G, there are up to two edges e 0 and e 00 in G f :
    1. if f (e) < cap(e), there is an edge e 0 = (v , w) in G f with residual capacity
    r (e 0 ) = cap(e) - f (e).
    2. if f (e) > 0, there is an edge e 00 = (w, v ) in G f with residual capacity r (e 00 ) =
    f (e).
\end{solution}

\end{parts}


% For tasks without simply remove the \begin{parts}...\part...\end{parts} commands