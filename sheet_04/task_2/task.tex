\question \textbf{Sampled Pointer (variable size)}

Given the same integer sequence from task 1

\begin{parts}
\part Compute the $\gamma$-code for each number and write down the bitvector $B$ representing the array $A$ in $\gamma$-coding % optional task text


\begin{solution}

\begin{align*}
B=&0001000 | 0001011 | 00000101000 | 00000110101 | 00000100000 | \\
 &00000100001 | 000010000 | 000010010
\end{align*}

\end{solution}

\part Write down the array $P$ of sampled pointers with $k = 3$ for the bitvector $B$.


\begin{solution}
\begin{align*}
    j=[1, 4, 7] \\
    P[j]=[1, 26, 49]
\end{align*}
\end{solution}

\part How do you access $A[2] = 11$ and $A[6] = 33$?


\begin{solution}
\begin{equation*}
    p = P\left[\lceil i/k \rceil\right]+\sum\limits_{t=\left(\lceil  i/k\rceil -1\right)k+1}^{i-1}l_t
\end{equation*}

For $A[2]=P[1]+\sum\limits_{t=1}^{1}l_t=1+7=8$.

For $A[6]=P[2]+\sum\limits_{t=4}^{5}l_t=26+11+11=48$. 

The $l_t$ values are not known so they have to be computed or the block has to be scanned starting from $P[\lceil i/k \rceil]$ until the numbers $\left(\lceil  i/k\rceil -1\right)k+1$ to $i-1$ are passed.


\end{solution}

\part How many bits are required for this data structure?

\begin{solution}

$76+\text{Bits for }P$. Assuming 8\,Bits per pointer: $76+3\cdot8=100$
\end{solution}

\end{parts}


% For tasks without simply remove the \begin{parts}...\part...\end{parts} commands