%Example of use of oxmathproblems latex class for problem sheets
\documentclass[12pt,a4paper,ngerman]{exam} 
\printanswers
\usepackage[ngerman]{babel}
\usepackage{amsmath}
\usepackage{amssymb}
\usepackage{graphicx}
\usepackage{algorithmicx}
\usepackage{algorithm}
\usepackage[noend]{algpseudocode}
\usepackage{fontspec}
\usepackage{fancyvrb}
\usepackage{tikz-qtree}
\usepackage{siunitx}
\usepackage{tikz}
\usepackage{csquotes}
\usepackage{stmaryrd}
\usetikzlibrary{arrows ,automata ,positioning}
\newcommand*\Let[2]{\State #1 $\gets$ #2}
\newcommand{\typen}{Janine Ponzer, Leonard Eckhoff, Mathis Börner}
\newcommand{\fach}{Foundations Computer Science WS22/23}
\renewcommand{\solutiontitle}{\noindent\textbf{Answer:}%
\enspace}
\pagestyle{headandfoot}
\firstpageheadrule
\runningheadrule
\firstpageheader{}{}{\typen \\ \fach}
\runningheader{}{}{\typen \\ \fach}
\firstpagefooter{}{}{\thepage\,/\,\numpages}
\runningfooter{}{}{\thepage\,/\,\numpages}


\DeclareSIUnit \basepair {bp}
\DeclareSIUnit\thomson{Th}

\begin{document}
\section*{Assignment 06}
\begin{questions}

\question \textbf{Bitvector compression}

\begin{parts}
\part Given the following Bitvector
\begin{align*}
    B = \left[11010010\,11010001\,01011111\,01110001\,10101000\,00101100\right]
\end{align*}
compute the compressed representation of $B$ with block length $b = 4$, as an array $C[1, ⌈n/b⌉]$, where $C[i] = c_i$ is the class that describes a block with i number of 1s, and an array $O[1, \lceil n/b\rceil]$, where $O[i] = o_i$ is the offset that identifies which element $i$ is among those of its class $c_i$.
When encoding the offsets, follow the technique presented in the lecture on slide 2055.

\begin{solution}

\includegraphics[width=0.8\linewidth]{task_2/a2_a.png}
\end{solution}


\part Decompress the following code into a Bitvector:
\begin{align*}
    b =4,\,C=[1,3,2,1],\,O=[0,3,2,2]
\end{align*}


\begin{solution}

\includegraphics[width=0.8\linewidth]{task_2/a2_b.png}
\end{solution}

\end{parts}


% For tasks without simply remove the \begin{parts}...\part...\end{parts} commands
\question \textbf{Bitvector compression}

\begin{parts}
\part Given the following Bitvector
\begin{align*}
    B = \left[11010010\,11010001\,01011111\,01110001\,10101000\,00101100\right]
\end{align*}
compute the compressed representation of $B$ with block length $b = 4$, as an array $C[1, ⌈n/b⌉]$, where $C[i] = c_i$ is the class that describes a block with i number of 1s, and an array $O[1, \lceil n/b\rceil]$, where $O[i] = o_i$ is the offset that identifies which element $i$ is among those of its class $c_i$.
When encoding the offsets, follow the technique presented in the lecture on slide 2055.

\begin{solution}

\includegraphics[width=0.8\linewidth]{task_2/a2_a.png}
\end{solution}


\part Decompress the following code into a Bitvector:
\begin{align*}
    b =4,\,C=[1,3,2,1],\,O=[0,3,2,2]
\end{align*}


\begin{solution}

\includegraphics[width=0.8\linewidth]{task_2/a2_b.png}
\end{solution}

\end{parts}


% For tasks without simply remove the \begin{parts}...\part...\end{parts} commands
\question \textbf{Bitvector compression}

\begin{parts}
\part Given the following Bitvector
\begin{align*}
    B = \left[11010010\,11010001\,01011111\,01110001\,10101000\,00101100\right]
\end{align*}
compute the compressed representation of $B$ with block length $b = 4$, as an array $C[1, ⌈n/b⌉]$, where $C[i] = c_i$ is the class that describes a block with i number of 1s, and an array $O[1, \lceil n/b\rceil]$, where $O[i] = o_i$ is the offset that identifies which element $i$ is among those of its class $c_i$.
When encoding the offsets, follow the technique presented in the lecture on slide 2055.

\begin{solution}

\includegraphics[width=0.8\linewidth]{task_2/a2_a.png}
\end{solution}


\part Decompress the following code into a Bitvector:
\begin{align*}
    b =4,\,C=[1,3,2,1],\,O=[0,3,2,2]
\end{align*}


\begin{solution}

\includegraphics[width=0.8\linewidth]{task_2/a2_b.png}
\end{solution}

\end{parts}


% For tasks without simply remove the \begin{parts}...\part...\end{parts} commands
%<task4>
%<task5>
%<task6>
%<task7>
%<task8>
%<task9>

\end{questions}
\end{document}
