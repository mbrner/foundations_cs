\question \textbf{Title}

\begin{parts}
\part Question % optional task text
Consider the two situations in a hash table of size m using open addressing with linear
probing:
1. You have n = m/2 keys in the table, where every even-indexed slot is occupied and
every odd- indexed slot is free.
2. You have n = m/2 keys in the table and the first n = m/2 locations are the ones
occupied.
Compute the average search cost for an unsuccessful search for both situations under the
assumption of simple uniform hashing.

\begin{solution}

Case 1: 
The average cost for an unsuccessful search is O(1). If the key hashes to an odd-indexed bucket, the bucket is empty and it is immedeatly clear, that the search is unsuccessful. 
If the key hashes to an even-indexed bucket, we just have to move one bucket until we reach the next empty bucket. Hence we make two moves, which is still in O(1).

Case 2: 
We compute the average cost of an unsuccessful search by adding the costs of the the empty half to the costs of the occupied half and dividing the term by m.
For the empty half we only have costs of 1 times m/2. For the occupied half the cost for the search is \(\sum_{i=0}^{m-1} \frac{(m/2)}{(m/2)-i}\).
Hence the average cost is \(\frac{m/2+\sum_{i=0}^{m/2-1} \frac{(m/2)}{(m/2)-i}}{m}\). 
This can be refactored to \(1/2+\sum_{i=1}^{m/2}{1/i}\). The latter term is the harmonic series. So the average runtime of lies in \(O(H_m)\), which is \(O(logn)\),
because \(H_n=log(n)+\gamma+1/2n+\epsilon_n\).



\end{solution}



\end{parts}


% For tasks without simply remove the \begin{parts}...\part...\end{parts} commands