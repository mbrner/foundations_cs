\question \textbf{Matching - Warmup}

\begin{parts}

\part What is maximum matching?
\begin{solution}
<<<<<<< HEAD
    Matching of the maximum cardinality or A matching, PP, of graph, GG, is said to be maximal if no other edges of GG can be added to PP because every node is matched to another node. 
        
=======
    Matching of maximum cardinality
>>>>>>> b9ba55c050d2beaf1870e4415d82907fcc0fb65d
\end{solution}

\part Is Matching using a directed or undirected graph?
\begin{solution}
<<<<<<< HEAD
    Matching is using an undirectedgraph. 
=======
    undirected
>>>>>>> b9ba55c050d2beaf1870e4415d82907fcc0fb65d
\end{solution}

\part What is perfect matching?
\begin{solution}
<<<<<<< HEAD
    Every vertex V is matched.
=======
    Every vertex in $V$ is matched.
>>>>>>> b9ba55c050d2beaf1870e4415d82907fcc0fb65d
\end{solution}

\part What is a bipartite graph?
\begin{solution}
<<<<<<< HEAD
A graph G = (V , E) is bipartite if there exist A, B $\subseteq$ V with A $\cup$ B = V , A $\cap$ B = $\emptyset$ and each edge in E has one end in A and one end in B
=======
A graph $G = (V , E)$ is bipartite if there exist $A, B \subset V$ with $A 	\cup B = V , A \cap B = \emptyset $ and
    each edge in $E$ has one end in $A$ and one end in $B$.
>>>>>>> b9ba55c050d2beaf1870e4415d82907fcc0fb65d
\end{solution}

\part Which pages of the \enquote{Algorithms 3rd.pdf} discusses matching?
\begin{solution}
<<<<<<< HEAD
985-1014
=======
Pages 732++ \enquote{26.3 Maximum bipartite matching}
>>>>>>> b9ba55c050d2beaf1870e4415d82907fcc0fb65d
\end{solution}

\part What is the difference between \enquote{maximal matching} and \enquote{maximum matching}?
\begin{solution}
<<<<<<< HEAD
    A matching is a maximum matching if it is a matching that contains the largest possible number of edges matching as many nodes as possible
       
=======
Maximum Matching is the collection of Maximum non-adjacent edges. Maximal Matching is the collection of minimum possible collection of non-adjacent edges.
>>>>>>> b9ba55c050d2beaf1870e4415d82907fcc0fb65d
\end{solution}

\end{parts}


% For tasks without simply remove the \begin{parts}...\part...\end{parts} commands